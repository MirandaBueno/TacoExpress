\documentclass{article}
\usepackage[utf8]{inputenc}
\usepackage{graphicx}      % include this line if your document contains figures
\usepackage{epstopdf}
\usepackage{xcolor, soul}
%\usepackage[margin=0.05in]{geometry}
\usepackage{cancel}
\usepackage{subfigure}
\usepackage{natbib}        % required for bibliography
\usepackage{amsmath,amssymb}
\usepackage{multirow}
\usepackage{color}
\usepackage{float}
\usepackage{booktabs}% http://ctan.org/pkg/booktabs
\newcommand{\tabitem}{~~\llap{\textbullet}~~}
\usepackage{tikz}
\usepackage{enumitem}
%\usepackage[shortlabels]{enumitem}

\usepackage[margin=1in]{geometry}
\usepackage[spanish,mexico]{babel}

\usepackage{graphicx}% http://ctan.org/pkg/graphicx
\usepackage{array}

\renewcommand{\thesection}{\Roman{section}} 
\renewcommand{\thesubsection}{\thesection.\Roman{subsection}}

%\usepackage[dvipsnames]{xcolor}
%%%%%%%%%%%%%%%%%%%%%%%%%%%%%%%%
% hipervinculos
\usepackage{hyperref}
\hypersetup{
    colorlinks=true,
    linkcolor=darkgray,
    filecolor=magenta,      
    urlcolor=blue,
    pdftitle={Overleaf Example},
    pdfpagemode=FullScreen,
}
\urlstyle{same}



%----------------CÓDIGO----------------%
\usepackage{color}
\definecolor{gray97}{gray}{.90}
\definecolor{gray75}{gray}{.40}
\definecolor{gray45}{gray}{.40}
\definecolor{vede}{RGB}{50,170,10}

\usepackage{listings}
\lstset{ frame=Ltb,
language=C++, %--------lenguaje
framerule=0pt,
aboveskip=.1cm,
framextopmargin=1pt,
framexbottommargin=1pt,
framexleftmargin=0.3cm,
tabsize=4,
xleftmargin=.05\textwidth, xrightmargin=.05\textwidth,
framesep=.1pt,
rulesep=.1pt,
backgroundcolor=\color{gray97},
rulesepcolor=\color{black},
%
stringstyle=\ttfamily,
showstringspaces = false,
%basicstyle=\small\ttfamily \color{red},
commentstyle=\color{vede},
keywordstyle=\color{red!80!black},
%
numbers=left,
numbersep=15pt,
numberstyle=\tiny,
basicstyle=\small\ttfamily \color{gray45},
numberfirstline = false,
breaklines=true,
%tabsize=5
}

% minimizar fragmentado de listados
\lstnewenvironment{listing}[1][]
{\lstset{#1}\pagebreak[0]}{\pagebreak[0]}

\lstdefinestyle{consola}
{basicstyle=\scriptsize\bf\ttfamily,
backgroundcolor=\color{gray75},
}


%%

\spanishdecimal{.}
%\usepackage{biblatex}
%\addbibresource{biblio.bib}


\begin{document}

%------------------- PORTADA -----------------------------

\begin{center}
    \begin{figure}[H]
    \centering
    \includegraphics[scale=0.2]{unam-escudo.png}
    \end{figure}
    \large Universidad Nacional Autónoma de México    \\
    \Large FACULTAD DE INGENIERÍA\\
    \rule{15cm}{0.05cm}
    \vspace{2cm}\\
%-----Titulo
    \huge \textbf{Proyecto final}\\
    \vspace{0.3cm}
%-----Subtitulo
    \Large Taco Express\\
    \vspace{0.5cm}
\end{center}
%-----Descripcion
   % \large Elaborar una presentación que explique mediante organizadores gráficos (mapas mentales, cuadros sinópticos, mapas conceptuales) o infografías el tema 1 Conceptos básicos de Bases de Datos.\\
    \vspace{1.5cm}
\begin{center}
  %  \large Equipo 2\\
    \Large \textbf{Miranda Bueno Fatima Yolanda}\\
    
    \vspace{0.2cm}
   % \large miranda.bueno1999@gmail.com\\
   \vspace{0.2cm}
 %   \large Ingeniería en Computación\\
    \vspace{0.2cm}
    \large 315337567\\
    \vspace{0.2cm}
    \large Grupo 1\\
    \vspace{0.2cm}
  % \large Equipo 2\\
    \vspace{2cm}
    \rule{15cm}{0.05cm}

    \vspace{1.5cm}
    \Large Computación Gráfica Avanzada\\
    \Large Ing. Reynaldo Martell Avila\\
    \vspace{0.9cm}
    \large  15 de Abril, 2024
    
    
\end{center}
\clearpage


\section{Objetivo}


\section{Proyecto en Github}

Se hizo seguimiento del proyecto con la plataforma Girhub.\\

\href{https://github.com/MirandaBueno/TacoExpress}{Link al proyecto}

\section{Calendario de actividades}

\begin{center} 
\begin{tabular}{|m{2cm}|m{2cm}|m{2cm}|m{2cm}|m{2cm}|m{2cm}|m{2cm}|} 
\multicolumn{7}{c}{Abril}\\
\hline
L&M&M&J&V&S&D\\
\hline
22&23&24&25&26&27&28\\
Toma de decisión del proyecto que se va a realizar: Taco Express&&Los modelos se van a crear con Blender, imágenes que se requieren se obtendrán de unsplash, y para los sonidos se usará OpenAL&&Inicio de la elaboración del proyecto final&Creación del proyecto en Visual Studio y se inicio el seguimiento del proyecto en Git y  \href{https://github.com/MirandaBueno/TacoExpress}{Github}&\\
\hline
\end{tabular} 
\end{center}

\begin{center} 
\begin{tabular}{|m{2cm}|m{2cm}|m{2cm}|m{2cm}|m{2cm}|m{2cm}|m{2cm}|} 
\multicolumn{7}{c}{Mayo}\\
\hline
L&M&M&J&V&S&D\\
\hline
29&30&01&02&03&04&05\\
%Creación de modelos 3D
%Carga de modelos
&&&&&&\\
\hline
06&07&08&09&10&11&12\\
%Programación inicial del juego (interacción básica entre modelos, como movimiento o posicionamiento)
%Implementación inicial de los controles de los modelo
&&&&&&\\
\hline
13&14&15&16&17&18&19\\
%Refinamiento de modelos y animaciones
%Programación avanzada, como colisione
&&&&&&\\
\hline
20&21&22&23&24&25&26\\
%Integración de sonidos
%Refinamiento final de modelos, programación, animación
y sonidos
%Detección y corrección de errores
&&&&&&\\
\hline
27&28&29&30&&&\\
%Pruebas exhaustivas
%Presentación del proyecto final


&&&&&&\\
\hline
\end{tabular} 
\end{center}

\section{Modelos a crear}
\begin{itemize}
    \item personaje principal
    \item 
\end{itemize}

\section{Recetas}
\begin{itemize}
    \item tacos
    \begin{itemize}
        \item tortilla: sacarla de la caja, ponerla en plato
        \item carne: sacar de caja, cortar en tabla, y freir, poner en tortilla 
        \item toppings: poner de l molde cebolla, cilantro y salsa
    \end{itemize}
    \item chilaquiles
    \begin{itemize}
        \item tortilla: sacar, cortar, freir
        \item salsa
        \item toppings: cebolla, crema
    \end{itemize}
    \item Tamales
    \begin{itemize}
        \item poner hoja de platano
        \item poner masa
        \item agregar relleno
        \item envolver
        \item poner en olla al vapor 
    \end{itemize}
    \item Pozole
    \begin{itemize}
        \item cortar carne
        \item poner caldo, maiz y carne en olla
    \end{itemize}
    \item agua 
    \begin{itemize}
        \item tomar vaso
        \item sabores: jamaica, horchata
    \end{itemize}
    \item 
\end{itemize}

\section{Conclusión}

%\section*{Referencias}



%\cite{op}

%\bibliographystyle{apalike}


%\bibliography{biblio}

% \begin{enumerate}
% \setcounter{enumi}{6}
%     \item 
% \end{enumerate}

%-----Imagenes
%\begin{figure}[H]
%   \centering
%   \includegraphics[scale=1.0, trim=0cm 0cm 0cm 0cm, clip]{a.png}
%   \caption{}
%\end{figure}


%-----Código
%\begin{lstlisting}
%\end{lstlisting}


%-----Tablas
%\begin{center} 
%\begin{tabular}{c|c} 
%\multicolumn{4}{|c|}{columna}
%\multirow{3}{4em}{renglon}
%&\\  
%\hline 
%\cline{3-8}
%\end{tabular} 
%\end{center}

%--- Hyperlinks
%\href{http://www.overleaf.com}{Something Linky}

\end{document}
